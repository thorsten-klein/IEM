\usepackage[utf8]{inputenc}
\usepackage[T1]{fontenc}
%\usepackage{pslatex}			%viel fetter
%\usepackage{mathptmx} 			%viel fetter
%\usepackage{ae,aecompl}		%minimal fetter
%\usepackage{lmodern}			%minimal fetter
%\usepackage{microtype}			%cm-super
\usepackage[english, ngerman]{babel} 	% German new language/hyphenation
\usepackage{setspace} 			% für Zeilenabstand
\usepackage{textgreek} 			%greeks letters senkrecht					???
\usepackage{scrpage2}			% Kopf / Fusszeile
\usepackage{geometry}			%Seitenränder
\usepackage{graphicx}			% Includegraphics
%\usepackage{lastpage} 			%Use LastPage
\usepackage{listings}			% Source Code integrieren
\usepackage{color}				% Farben
\usepackage[colorlinks,
	pdfpagelabels=true,
	linkcolor = black,
	%plainpages = false,
	%hypertexnames = false,
	citecolor = black
]{hyperref}						% TOC verlinken
\usepackage{tocstyle}			% weniger Abstand im TOC
\usepackage{scrhack}			% Fehler von listings ausblenden
\usepackage{pdfpages}			% PDF includieren
\usepackage{csquotes}

\usepackage{caption}
\captionsetup{font=small}
\captionsetup[figure]{labelfont=bf,textfont=it}


\usepackage{hyphenat}
\sloppy 





\usepackage[
backend=biber,
%style=authoryear-icomp,    % Zitierstil
%style=verbose-trad2,
%style=authoryear,
%style=numeric,
style=alphabetic,
isbn=true,                % ISBN
pagetracker=true,          % ebd. bei wiederholten Angaben (false=ausgeschaltet, page=Seite, spread=Doppelseite, true=automatisch)
%maxbibnames=50,            % maximale Namen, die im Literaturverzeichnis angezeigt werden (ich wollte alle)
%maxcitenames=3,            % maximale Namen, die im Text angezeigt werden, ab 4 wird u.a. nach den ersten Autor angezeigt
%autocite=inline,           % regelt Aussehen für \autocite (inline=\parancite)
%block=space,               % kleiner horizontaler Platz zwischen den Feldern
backref=true,              % Seiten anzeigen, auf denen die Referenz vorkommt
backrefstyle=three+,       % fasst Seiten zusammen, z.B. S. 2f, 6ff, 7-10
%date=short,                % Datumsformat
]{biblatex}

\setlength{\bibitemsep}{1em}     % Abstand zwischen den Literaturangaben
\setlength{\bibhang}{2em}        % Einzug nach jeweils erster Zeile

\bibliography{literature}  % Bibtex-Datei wird schon in der Preambel eingebunden 
\nocite{*}

